\section{Introduction}\label{introduction}

\begin{frame}{What is a Benchmark}

I haven't found a clear definition of what a benchmark is. However, here
is what I think makes a \emph{numerical example} a benchmark

\begin{enumerate}
\def\labelenumi{\arabic{enumi}.}
\tightlist
\item
  \textbf{Common acceptance as a benchmark} -- there are other
  publications that discuss the same setup.
\item
  \textbf{Practical relevance} -- either in applications or as a testing
  field for numerical algorithms.
\item
  \textbf{Reliable reference data} -- so that others can test their
  codes and methods against it.
\end{enumerate}

\end{frame}

\begin{frame}{What is a Benchmark}

\begin{quote}
Basically, everything that would motivate a fellow researchers to use
the provided setup and data to \emph{benchmark} their code.
\end{quote}

\end{frame}

\begin{frame}{Fluid Structure Interaction}

\begin{figure}
\centering
\includegraphics{pics/fsi2.gif}
\caption{Example of a cylinder with a tail}
\end{figure}

\begin{itemize}
\tightlist
\item
  Changing domain.
\item
  Coupling of Models (and scales).
\end{itemize}

\end{frame}

\section{The model}\label{the-model}

\begin{frame}{Verbose}

\begin{itemize}
\tightlist
\item
  A fluid flows through a channel with a sphere that can rotate freely.
\item
  The stresses at the sphere/fluid interface induce rotation.
\item
  The \emph{no-slip} condition induces motion of the flow at the
  interface.
\end{itemize}

\end{frame}

\begin{frame}

\begin{block}{The flow}

\begin{equation*}
        \rho_f\left(\partial_t v + (v \cdot\nabla)v \right) - \nabla \cdot \sigma(v ,p) = 0, \quad \nabla\cdot v  = 0,
\end{equation*}

with the stress-tensor

\begin{equation*}
    \sigma (v,p) = \rho _ f\nu\left( \nabla v+\nabla v^T \right) - p I
\end{equation*}

and with standard boundary conditions and in particular

\begin{equation*}
    v = v_s, \quad \text{on } \mathcal I,
\end{equation*}

where \(v_s\) is the solid's velocity at the fluid-solid interface.

\end{block}

\end{frame}

\section{Implementation}\label{implementation}

\begin{frame}{Code Base}

There were 5 independent implementations using established libraries:

\begin{itemize}
\tightlist
\item
  \href{https://ngsolve.org/}{Netgen/NGSolve}
\item
  \href{https://fenicsproject.org/download/}{FEniCS/dolfin}
\item
  Gascoigne
\item
  \href{https://www.scipy.org}{SciPy}
\end{itemize}

\end{frame}

\begin{frame}{Code Availability}

Full data sets for the results as well as all implementations can be
found at

\begin{quote}
\href{https://doi.org/10.5281/zenodo.3253455}{DOI:
10.5281/zenodo.3253455}
\end{quote}

\end{frame}

\section{Conclusion}\label{conclusion}

\begin{frame}{References}

\hypertarget{refs}{}
\hypertarget{ref-WahRLHM19}{}
von Wahl, Henry, Thomas Richter, Christoph Lehrenfeld, Jan Heiland, and
Piotr Minakowski. 2019. ``Numerical Benchmarking of Fluid-Rigid Body
Interactions.'' \emph{Computers \& Fluids}.
doi:\href{https://doi.org/10.1016/j.compfluid.2019.104290}{10.1016/j.compfluid.2019.104290}.

\end{frame}
